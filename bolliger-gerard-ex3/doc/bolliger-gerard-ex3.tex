\documentclass[11pt]{article}

\usepackage{amsmath}
\usepackage{amssymb} 
\usepackage{textcomp}
\usepackage{float}
\usepackage[top=0.8in, bottom=0.8in, left=0.8in, right=0.8in]{geometry}
\usepackage{graphicx}
\usepackage{wrapfig}
\graphicspath{ {./} }
% Add other packages here %


% Put your group number and names in the author field %
\title{\bf Excercise 3\\ Implementing a deliberative Agent}
\author{Group \textnumero : 90  Kyle Gerard, Yann Bolliger}


% N.B.: The report should not be longer than 3 pages %


\begin{document}
\maketitle

\section{Model Description}

\subsection{Intermediate States}
Mathematically seen our state contains all positions of all the objects on the 
map. That means the current position of the agent, all the tasks the agent has already picked up
and all the tasks to be picked up yet. This could be written as the following state space:

$$
\mathcal{S} = \mathcal{C} 
\times
\{ A_{carried} : A_{carried} \subseteq \mathcal{T} \}
\times
\{ A_{pending} : A_{pending} \subseteq \mathcal{T} , 
  A_{pending}  \cap A_{carried}  = \emptyset
  \}
$$

where $\mathcal{C} $ is the set of cities and $ \mathcal{T}$ is the overall set 
of tasks. 

\subsection{Goal State}
A goal state is reached when all packets/tasks are delivered. Therefore it is 
defined as:

$$
\mathcal{S}_{goal} =  \{  (C, A_{carried}, A_{pending}) \in  \mathcal{S} : 
A_{carried} = \emptyset \wedge A_{pending} = \emptyset \} \subset  \mathcal{S}
$$

\subsection{Actions}
At each state there are up to three possible actions: move, pickup, deliver. 
They all have some conditions on the current state:

\begin{itemize}
  \item
  The agent can move only to neighbors of the current city encoded in the state. 
  This changes the city for the next state, the rest stays the same.
  
    \item
  The agent can deliver a task if $ \exists T \in A_{carried}$ which has 
  to be delivered at the current city of the agent.
  Then this task is removed from $A_{carried}$ for the next state.
  
  \item
  The agent can pickup a task only if $\exists T \in A_{pending} $ which has to 
  be picked up from the current city of the agent and if 
  $\sum_{T \in A_{carried} } \text{weight} (T) < $ maximal capacity of the 
  vehicle.
  In that case $T$ is removed from  $A_{pending}$  and added to $A_{carried}$ 
  for the next state.
\end{itemize}



\section{Implementation}

\subsection{BFS}
We closely followed the BFS algorithm. However, instead of stopping the 
search at the first goal node, we traverse the entire tree and take the plan 
with the minimal cost at the end. 

In order to make this memory consuming search fast, there are some 
important keypoints. We defined a specialised node class that acts as nodes in 
the search tree. These nodes don't only store the formal \texttt{State} of the 
agent but also the plan up to this state and the corresponding cost. This 
saves us from recomputing the entire two properties for each search node.

Another small point is cylce detection. This is where we prevent reconsidering 
states we have already seen unless we find them with a smaller cost.

The most important speedup comes however from choosing the correct data-structures.
For $\mathcal{O}(1)$ lookups of states for cycle detection and other lookups we 
always use a \texttt{HashMap} together with the hash function of the class 
\texttt{State}.
For the search agenda queue we rather use linked list in order to prevent costly 
array copying. This makes our BFS reasonably fast for up to XXXX tasks.

\subsection{A*}
As in BFS, we closely followed to algorithm given and made use of an admissible 
heuristic. This lets us do simple cycle detection because we know that we find a 
state optimally in the first place. Again, we wrote a specialised \texttt{Node} 
class that keeps track of the state, the plan, the cost and the heuristic in the 
search tree.

We also reused \texttt{HashMap} for cycle detection but here,
 we leveraged even more of Java's rich collection API. In fact, the search 
agenda queue is a \texttt{PriorityQueue}. This is a binary heap implementation 
and works with the \texttt{compareTo} metod of \texttt{Node}. Therefore it automatically 
gives us the search node with the lowest $f()$ value in $\mathcal{O}(\log n)$ 
time. Which is an immense advantage compared to the approach where the entire 
list of search nodes is sorted in $\mathcal{O}(n \log n)$ time!

\subsection{Heuristic Function}
The heuristic in our A*-Algorithm must be admissible. It is defined as:
$$
h(S = (C, A_{carried}, A_{pending})) = \max (h(A_{carried}), h(A_{pending})) 
$$$$
h(A_{carried}) = \max_{T \in A_{carried}} \text{distance}(C, \text{destination}(T))
$$$$
h(A_{pending}) = \max_{T \in A_{pending}} \text{distance}(C, \text{origin}(T)) + 
\text{distance}(\text{origin}(T), \text{destination}(T))
$$

This is admissible because if the agent has at least one pending task, it has \textit{at least}
to go to the task's origin, pick it up and deliver it. In that case the heuristic exactly 
calculates the cost. If the agent has more tasks, the heuristic will underestimate the cost. 
The same is true if the agent only has one carried task. Therefore the heuristic 
is admissible.

--> monotonic, optimal?


\section{Results}

\subsection{Experiment 1: BFS and A* Comparison}
% Compare the two algorithms in terms of: optimality, efficiency, limitations %
% Report the number of tasks for which you can build a plan in less than one minute %

\subsubsection{Setting}
% Describe the settings of your experiment: topology, task configuration, etc. %

\subsubsection{Observations}
% Describe the experimental results and the conclusions you inferred from these results %


\subsection{Experiment 2: Multi-agent Experiments}
% Observations in multi-agent experiments %

\subsubsection{Setting}
% Describe the settings of your experiment: topology, task configuration, etc. %

\subsubsection{Observations}
% Describe the experimental results and the conclusions you inferred from these results %

\end{document}