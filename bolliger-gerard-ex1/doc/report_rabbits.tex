\documentclass[11pt]{article}

\usepackage{amsmath}
\usepackage{textcomp}
\usepackage[top=0.8in, bottom=0.8in, left=0.8in, right=0.8in]{geometry}
% Add other packages here %



% Put your group number and names in the author field %
\title{\bf Excercise 1.\\ Implementing a first Application in RePast: A Rabbits Grass Simulation.}
\author{Group \textnumero: 90  Kyle Gérard, Yann Bolliger}

\begin{document}
\maketitle

\section{Implementation}

\subsection{Assumptions}
\begin{itemize}

\item 
Our grass growth is bounded above linearly. We choose \texttt{grassGrowthRate} 
squares at each iteration tick and put grass on them. Regardless of whether 
there has already been grass or not. In that manner we add \textit{at most} \texttt{grassGrowthRate} 
new squares of grass at each tick.

\item 
The energy of the grass in a cell is always equal to \texttt{grassEnergy}.
This can be adjusted as a parameter.

 
 - rabbits can be born on grass
 - if rabbits can't move, they don't move but they still loose energy (1)
 - rabbits can go back and forth
 - default birth energy 30
 
\end{itemize}



\subsection{Implementation Remarks}
% Provide important details about your implementation, such as handling
of boundary conditions %

\section{Results}
% In this section, you study and describe how different variables
(e.g. birth threshold, grass growth rate etc.) or combinations of variables
influence the results. Different experiments with diffrent settings are
described below with your observations and analysis

\subsection{Experiment 1}

\subsubsection{Setting}

\subsubsection{Observations}
% Elaborate on the observed results %

\subsection{Experiment 2}

\subsubsection{Setting}

\subsubsection{Observations}
% Elaborate on the observed results %

\vdots

\subsection{Experiment n}

\subsubsection{Setting}

\subsubsection{Observations}
% Elaborate on the observed results %

\end{document}