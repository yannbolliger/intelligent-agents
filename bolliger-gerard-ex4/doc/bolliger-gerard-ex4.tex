\documentclass[11pt]{article}

\usepackage{amsmath}
\usepackage{amssymb}
\usepackage{textcomp}
\usepackage[top=0.8in, bottom=0.8in, left=0.8in, right=0.8in]{geometry}
% Add other packages here %


% Put your group number and names in the author field %
\title{\bf Excercise 4\\ Implementing a centralized agent}
\author{Group \textnumero : 90  Kyle Gerard, Yann Bolliger}


% N.B.: The report should not be longer than 3 pages %


\begin{document}
\maketitle

\section{Solution Representation}

\subsection{Variables} 

Instead of representing a vehicle's journey as a sequence of tasks, we chose to
represent it as a sequence of $pickup$ and $delivery$ actions. Each task $t \in
\mathcal{T}$ has one action of both types (\ref{eq:1}). This accounts for the
fact that a vehicle can carry multiple tasks at a time if there are two pickups
in a row. 
The variables (\ref{eq:2}) define the first pickup of each vehicle
$v \in \mathcal{V}$. If the variable is $\mathtt{null}$ this means the vehicle 
does not accomplish any actions.
\begin{eqnarray} 
\label{eq:1}
\mathcal{P} = \{pickup(t) : t \in \mathcal{T}\} , \;  
\mathcal{D} = \{delivery(t) : t \in \mathcal{T}\} , \; 
\mathcal{A} =  \mathcal{P} \cup \mathcal{D}
\\
\label{eq:2}
\forall v \in \mathcal{V}  : \;
firstPickup(v)  \in \mathcal{P} \cup \{\mathtt{null}\} 
\\
\label{eq:3}
\forall a \in \mathcal{A}  : \;
nextAction(a) \in \mathcal{A}  \cup \{\mathtt{null}\} 
\\
\label{eq:4}
\forall a \in \mathcal{A}  :
vehicle(a) \in \mathcal{V};
\;\;
\forall a \in \mathcal{A}  :
time(a) \in \mathbb{N}
\end{eqnarray}

A vehicles journey is completely defined by its $firstPickup$ and the variables
(\ref{eq:3}) where again the $\mathtt{null}$ signifies that a vehicle has no
further actions to perform. We will call the sequence of actions of a vehicle
its action chain. All travels are made on the shortest possible path.

The variables (\ref{eq:4}) help us state the constraints. The $vehicle$
variables define which vehicle carries out a certain action. This can be derived
from $firstPickup$ at the start of the action chain defined by (\ref{eq:3}). The
second variable can also be derived from the action chains. It simply gives the
rank of each action in the chain. Both derivations are more formally stated in
the next paragraph.

\subsection{Constraints}
As explained before, the action chain of a vehicle defines the $time$ 
and the $vehicle$ for each action:
\begin{eqnarray}
firstPickup(v) = a \Rightarrow vehicle(a) = v;  &
nextAction(b)  = c \Rightarrow vehicle(c) = vehicle(b)
\\
firstPickup(v) = a \Rightarrow time(a) = 1;  &
nextAction(b)  = c \Rightarrow time(c) = time(b) + 1
\end{eqnarray}

Additionally, the same vehicle must $pickup$ and $deliver$ a task (\ref{eq:v}). 
It has to pickup the task before it delivers it (\ref{eq:t})
and \textbf{each task must be picked up and delivered}.
\begin{eqnarray}
\forall a \in \mathcal{A}  :
nextAction(a) \neq a
\\
nextAction(a) = \mathtt{null} \Rightarrow a \in \mathcal{D}
\text{ and } nextAction(\mathtt{null}) = \mathtt{null}
\\
\label{eq:v}
\forall t \in \mathcal{T}: 
vehicle(pickup(t)) = vehicle(delivery(t)) 
\\
\label{eq:t}
\forall t \in \mathcal{T}: 
time(pickup(t)) < time(delivery(t))
\\
\forall t \in \mathcal{T} \; 
\exists  \, \{ pickup(t), delivery(t) \} \subset \mathcal{A} \;\;
\\
\forall a \in \mathcal{A} \;
\exists \, v \in \mathcal{V} : 
vehicle(a) = v
\end{eqnarray}

Last but not least, at all times $\tau$ a vehicle $v$ can never 
carry more weight than its capacity.
$$
carriedTasks(\tau, v) = \{t \in \mathcal{T}: 
vehicle(pickup(t)) = v \wedge 
time(pickup(t)) < \tau \wedge 
time(delivery(t)) > \tau \}
$$$$
\forall \tau \in \mathbb{N}, 
\forall v \in \mathcal{V}:
\sum_{t \, \in \, carriedTasks(\tau, v)} weight(t) \leq
capacity(v)
$$

\subsection{Objective function} 

The goal of the company is to maximise the reward. Because all tasks have to be
delivered, all rewards will be earned and the overall reward is constant. Thus
the objective function we want to minimise is the cost of the overall assignment
$\mathcal{S}$. We define $dist(a,b)$ to be the shortest distance between the
associated cities of actions $a,b \in \mathcal{A}$, $dist(a, \mathtt{null}) =
0$, $start(v)$ is the initial postion and $cost(v)$ the cost per kilometre of
vehicle $v$.
$$
cost(\mathcal{S}) = 
\sum_{v \in \mathcal{V}} dist(start(v), firstPickup(v))
\cdot cost(v)
 +
\sum_{a \in \mathcal{A}} dist(a, nextAction(a))
\cdot cost(vehicle(a))
$$


\section{Stochastic optimization}

\subsection{Initial solution}

Our initial solution is already a valid, greedy solution. Therefore, our program
returns a correct assignment at all times, even if there is no computation time.
The initial solution is computed by appending consecutively the $pickup$ and
$delivery$ pair of each task to the action chain of the vehicle that is the
closest (and has enough capacity). This distance is calculated between the last
position in the action chain of each vehicle and the $pickup$ location.
Therefore in the initial solution the vehicles don't carry multiple tasks at
once.


\subsection{Generating neighbours}

We generate neighbors of a current assignment $\mathcal{S}$ by applying two
stochastic operators. For both of them, we randomly choose a vehicle $v$. Then,
for the first set of neighbors, we remove both actions $p, d$ for a random task
$t$ from the action chain of $v$. The neighbors result from inserting $p, d$
into the action chains of all other neighbors. This yields a lot of
possibilities because the $p, d$ can be inserted in many ways into the new
action chain as long as the capacity and the $p$ before $d$ constraint is
respected. The second set of neighbors is obtained from reordering the actions
in $v$'s chain without removing or inserting. Still the reordering has to
respect the capacity and the $p$ before $d$ constraint.


\subsection{Stochastic optimization algorithm}

As proposed in the paper\footnote{Radu Jurca, Nguyen Quang Huy and Michael
Schumacher {\em Finding the Optimal Delivery Plan: Model as a Constraint
Satisfaction Problem} 2006-2007: Intelligent Agents course} we use stochastic
local search to find a better solution than the initial one. Thereby we generate
neighbors at each iteration and with probability $p$, we take the least costly
neighbor as a new solution.

The critical addition we made to the algorithm avoids getting stuck in local
minima. In fact, each time a new solution is chosen, we also add this solution
to a set called \texttt{formerSolutions}. The algorithm is not allowed to
subsequently choose a solution with the \textbf{same cost} as one that is
already in the set. Therefore it has to keep exploring new solutions with new
and possibly lower cost. At the end the overall minimum that was ever visited is
returned.


\section{Results}

\subsection{Experiment 1: Model parameters}
% if your model has parameters, perform an experiment and 
% analyze the results for different parameter values %

\subsubsection{Setting} 
% Describe the settings of your experiment: topology,
% task configuration, number of tasks, number of vehicles, etc. % 
% and the parameters you are analyzing %

\subsubsection{Observations}
% Describe the experimental results and the conclusions you inferred 
% from these results %

\subsection{Experiment 2: Different configurations}
% Run simulations for different configurations of the environment 
% (i.e. different tasks and number of vehicles) %

\subsubsection{Setting}
% Describe the settings of your experiment: topology, task 
% configuration, number of tasks, number of vehicles, etc. %

\subsubsection{Observations}
% Describe the experimental results and the conclusions you inferred 
% from these results. Reflect on the fairness of the optimal plans. 
% Observe that optimality requires some vehicles to do more work 
% than others. How does the complexity of your algorithm depend on the 
% number of vehicles and various sizes of the task set? %

\end{document}