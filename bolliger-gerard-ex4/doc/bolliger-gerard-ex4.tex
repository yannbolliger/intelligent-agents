\documentclass[11pt]{article}

\usepackage{amsmath}
\usepackage{amssymb}
\usepackage{textcomp}
\usepackage[top=0.8in, bottom=0.8in, left=0.8in, right=0.8in]{geometry}
% Add other packages here %


% Put your group number and names in the author field %
\title{\bf Excercise 4\\ Implementing a centralized agent}
\author{Group \textnumero : 90  Kyle Gerard, Yann Bolliger}


% N.B.: The report should not be longer than 3 pages %


\begin{document}
\maketitle

\section{Solution Representation}

\subsection{Variables} 

Instead of representing a vehicle's journey as a sequence of tasks, we chose to
represent it as a sequence of $pickup$ and $delivery$ actions. Each task 
$t \in \mathcal{T}$
has one action of both types.

$$
\mathcal{P} = \{pickup(t) : t \in \mathcal{T}\} , \;  
\mathcal{D} = \{delivery(t) : t \in \mathcal{T}\} , \; 
\mathcal{A} =  \mathcal{P} \cup \mathcal{D}
$$

This accounts for the fact that a vehicle can carry multiple tasks at a time if
there are two pickups in a row. The following variables define the 
first pickup of each vehicle (where $\mathcal{V}$ is the set vehicles). 

$$
\forall v \in \mathcal{V}  : \;
firstPickup(v)  \in \mathcal{P} \cup \{\mathtt{null}\} 
$$

If the variable is $\mathtt{null}$ this means the vehicle does not accomplish
any actions. All subsequent actions of a vehicle are defined by the next set of
variables:

$$
\forall a \in \mathcal{A}  : \;
nextAction(a) \in \mathcal{A}  \cup \{\mathtt{null}\} 
$$

where again the $\mathtt{null}$ signifies that a vehicle has no further actions
to perform. We will define two other sets of variables which will clarify the
former:

$$
\forall a \in \mathcal{A}  :
vehicle(a) \in \mathcal{V};
\;\;
\forall a \in \mathcal{A}  :
time(a) \in \mathbb{N}
$$

The $vehicle$ variables define which vehicle carries out a certain action.
This can be derived from the $firstPickup$ action at the start of
each action chain defined by $nextAction$. (For example if
$firstPickup(v) = a, nextAction(a) = b$ then 
$vehicle(a) = vehicle(b) = v$.)

The second variable can also be 
derived from the action chains. It simply gives the rank of each action
in the chain (for example if 
$firstPickup(v) = a, nextAction(a) = b$ then $time(a) = 1$, $time(b) = 2$).

\subsection{Constraints}
As explained before, the action chain for each vehicle define the $time$ 
and the $vehicle$ for each action:

$$
firstPickup(v) = a \Rightarrow time(a) = 1; \;\;
nextAction(b) = c \Rightarrow time(c) = time(b) + 1
$$
$$
firstPickup(v) = a \Rightarrow vehicle(a) = v; \;\;
nextAction(b) = c \Rightarrow vehicle(c) = vehicle(b)
$$

Additionally, the same vehicle must $pickup$ and $deliver$ a task. 
It has to pickup the task before it delivers it, of course, 
and \textbf{each task must be picked up and delivered}:
$$
\forall a \in \mathcal{A}  :
nextAction(a) \neq a, \,
nextAction(a) = \mathtt{null} \Rightarrow a \in \mathcal{D}
$$
$$
\forall t \in \mathcal{T}: 
vehicle(pickup(t)) = vehicle(delivery(t)) \in \mathcal{V}
$$
$$
\forall t \in \mathcal{T}: 
time(pickup(t)) < time(delivery(t))
$$
$$
\forall t \in \mathcal{T} \; 
\exists  \, \{ pickup(t), delivery(t) \} \subset \mathcal{A} \;\;
\text{and} \;\;
\forall a \in \mathcal{A} \;
\exists \, v \in \mathcal{V} : 
vehicle(a) = v
$$

Last but not least, at each time $\tau$ a vehicle $v$ can never 
carry more weight than its capacity.
$$
carriedTasks(\tau, v) = \{t \in \mathcal{T}: 
vehicle(pickup(t)) = v \wedge 
time(pickup(t)) < \tau \wedge 
time(delivery(t)) > \tau \};
$$
$$
\forall \tau \in \mathbb{N}:
\sum_{t \, \in \, carriedTasks(\tau, v)} weight(t) \leq
capacity(v)
$$



\subsection{Objective function}
The goal of the company/agent is to maximise the reward. Because all tasks 
have to be delivered, all rewards will be earned and the overall reward
sum is therefore constant. Thus the objective function we want to minimise 
is the cost of the overall assignment $\mathcal{S}$.


\section{Stochastic optimization}

\subsection{Initial solution}
% Describe how you generate the initial solution %

\subsection{Generating neighbours}
% Describe how you generate neighbors %

\subsection{Stochastic optimization algorithm}
% Describe your stochastic optimization algorithm %


\section{Results}

\subsection{Experiment 1: Model parameters}
% if your model has parameters, perform an experiment and 
% analyze the results for different parameter values %

\subsubsection{Setting} 
% Describe the settings of your experiment: topology,
% task configuration, number of tasks, number of vehicles, etc. % 
% and the parameters you are analyzing %

\subsubsection{Observations}
% Describe the experimental results and the conclusions you inferred 
% from these results %

\subsection{Experiment 2: Different configurations}
% Run simulations for different configurations of the environment 
% (i.e. different tasks and number of vehicles) %

\subsubsection{Setting}
% Describe the settings of your experiment: topology, task 
% configuration, number of tasks, number of vehicles, etc. %

\subsubsection{Observations}
% Describe the experimental results and the conclusions you inferred 
% from these results. Reflect on the fairness of the optimal plans. 
% Observe that optimality requires some vehicles to do more work 
% than others. How does the complexity of your algorithm depend on the 
% number of vehicles and various sizes of the task set? %

\end{document}